\documentclass[11pt]{article}
\usepackage{tikz}
\usepackage{amsmath}
\usepackage{mathtools}
\usepackage{tabularx}

%\usepackage{xfrac}
%\usepackage{hyperref}
%\usepackage[export]{adjustbox}
\def\checkmark{\tikz\fill[scale=0.4](0,.35) -- (.25,0) -- (1,.7) -- (.25,.15) -- cycle;} 
\usepackage{proj} 	% pull in style header
\usepackage{array}
\usepackage{sectsty}
\usepackage{soul}
\usepackage{float}
\restylefloat{table}

\lhead{ECE582: Formal Verification}


%----------------------------------------------------------------------------------------
%	TITLE SECTION
%----------------------------------------------------------------------------------------


\newcommand{\horrule}[1]{\rule{\linewidth}{#1}} % Create horizontal rule command with 1 argument of height

\title{	
\normalfont \normalsize 
\textsc{\LARGE Portland State University}\\[1.5cm] % Name of your university/college
\textsc{\Large Project 2}\\[0.5cm] % Major heading such as course name
\textsc{\large ECE582}\\[0.5cm] % Minor heading such as course title
%\textsc{Portland State University} \\ [25pt] % Your university, school and/or department name(s)
\horrule{1.2pt} \\[0.4cm] % Thin top horizontal rule
\huge Model Checking by NuSMV \\ % The assignment title
\horrule{1.2pt} \\[0.5cm] % Thick bottom horizontal rule
}

%----------------------------------------------------------------------------------------
%	AUTHOR SECTION
%----------------------------------------------------------------------------------------


\begin{document}\raggedright
\author{Erik Rhodes \and Jordan Fluth} % Your name
\maketitle % Print the title
\thispagestyle{empty}
\cfoot{\textit{Page \thepage { of} \pageref{LastPage}}}
\lhead{ECE582}
\chead{Project 2}
\rhead{Erik Rhodes \& Jordan Fluth}


\begin{figure}[h]\centering
\includegraphics[height=0.65\textwidth]{images/model.jpg}
	%\caption{Gameplay Block Diagram}
		\label{LED}
	\end{figure}
	
\newpage


%Symbols \And \wedge \vee \neg \to \gets \iff


\section{Combinational Lock}
The combinational lock was modeled in NuSMV, using the operational parameters defined in the project description, in order to verify the properties below. We hypothesized that the five 
properties below were true for the combinational lock system. The results of the NuSMV simulation indicate that our properties were in fact true. 
	
	\subsection{Added properties}
	\textbf{Property 1:} AG((open * ~up * ~down) $\to$ (open)) 	\\	
	\textbf{Description:} If the lock is open and it is not interrupted, it can remain open indefinitely. \\
	\vspace{12pt}
	\textbf{Property 2:} AG(state=1 * ~open * down  * ~up * position=21) $\to$ (state=2)\\ 
	\textbf{Description:} A transition to state two can only happen under the required circumstances. \\
	\vspace{12pt}
	\textbf{Property 3:} AG((open * up * ~down) $\vee$  (open * down * ~up)) $\to$ AX(~open)\\ \textbf{Description:} If the lock is open and it is interrupted the the lock will close. \\
	\vspace{12pt}
	\textbf{Property 4:} AG(EF(~open)) \\
	\textbf{Description:} It is always possible for the lock to be closed.\\
	\vspace{12pt}
	\textbf{Property 5:} AG(down * position=15) $\to$ (~open) \\
	\textbf{Description:} The lock will not open by counting down to 15.\\

	
	\subsection{Results}
	%TODO remove line numbers?
 \begin{lstlisting}[caption=NuSMV Output, label=p1_out]		
	-- specification (AG ((open & !up) & !down) -> open)  is true
	-- specification (AG ((((state = left_12 & !open) & down) & !up) & position = 21) -> state = right_21)  is true
	-- specification (AG (((open & up) & !down) | ((open & down) & !up)) -> AX !open)  is true
	-- specification AG (EF !open)  is true
	-- specification (AG (down & position = 15) -> !open)  is true

  \end{lstlisting}
	


\subsection{Proofs}
The properties mentioned above are explained in figure \ref{p1_p}.

	\begin{figure}[h]\centering
	\includegraphics[height=0.45\textwidth]{images/p1_proof.png}
	\caption{Diagram showing Lock States}
		\label{p1_p}
	\end{figure}

\begin{enumerate}
\item In state three the lock is open and if neither up, nor down is asserted then the lock continues to stay in state three and remain open.
\item In order for a transition to state two, you must have been in state one, the position of the count must be at 21, and you must have reached 21 by counting down.
\item In state three the lock is open, the only way to transition from state three, and thus close the lock, is by moving the position count up or down. If either of these happen there will be a transition to state zero and the lock will be closed.
\item In states zero, one, and two the lock is closed. State three is the only state in which the lock is open. Since there is a transition out of state three that is possible then there exist the possibility for the lock to be closed.
\item If in state two and counting up occurs there will be a transition to state zero. Therefore the lock will not be opened. 
\end{enumerate}	

	
%\begin{figure}[h!]\centering
%\includegraphics[height=0.55\textwidth]{images/c1c2.PNG}
%	\caption{Miter Circuit of C1 and C2}
%		\label{c2}
%	\end{figure}

\section{Kripke Structures} 

\subsection{Structure 1}
	\subsubsection{Added properties}
	
		\textbf{Property 1:} 	AG(state=3) $\to$ AX(state=4) 	\\	
		\textbf{Description:} From state three, the next transition must be to state four. \\
		\vspace{12pt}
		
		\textbf{Property 2:} AG(state=4) $\to$ AX(state=2)\\ 
		\textbf{Description:} From state four, the next transition must be to state two. \\
		\vspace{12pt}
		
		\textbf{Property 3:} 	AG(state=5) $\to$ EG(state=5)\\ 
		\textbf{Description:} From state five, it is possible to remain in state five indefinitely. \\
		\vspace{12pt}
		
		\textbf{Property 4:} AG(state=1) $\to$ ~EG(state=5) \\
		\textbf{Description:} From state one, the next state cannot be five.\\
		\vspace{12pt}
		
		\textbf{Property 5:} AG(state=2 | state=5) $\to$ AX(state=3)  \\
		\textbf{Description:} State three is reachable by state two or state five.\\
	
	
	\subsubsection{Results}
	
	\begin{lstlisting}[caption=NuSMV Kripke Diagram Output, label=p2_1_out]	
	-- specification (AG state = 3 -> AX state = 4)  is true
	-- specification (AG state = 4 -> AX state = 2)  is true
	-- specification (AG state = 5 -> EG state = 5)  is true
	-- specification (AG state = 1 -> !(EG state = 5))  is true
	-- specification (AG (state = 2 | state = 5) -> AX state = 3)  is true
	\end{lstlisting}
	
	\subsubsection{Proofs}
	The properties mentioned above are explained in figure \ref{p2_1_p}.
	\begin{figure}[h]\centering
	\includegraphics[height=0.45\textwidth]{images/p2_1_proof.png}
		\caption{State Diagram Explaining Results}
			\label{p2_1_p}
		\end{figure}
\begin{enumerate}
\item If in state three there is only one possible transition. When in state three the next state will be state four.
\item  If in state four there is only one possible transition. When in state four the next state will be state two.
\item  If in state five one possible transition loops back to state five, therefore once you get in state five it is always possible to remain in state five.
\item  If in state one, there is no transition to state five. Therefore, there is no path that leads from state one, to state five as the next state.
\item  If in state two, or state five, there is a possible transition to state three.
\end{enumerate}	

	
\subsection{Figure 2}
	\subsubsection{Added properties}
	
		\textbf{Property 1:} 	AG(state=6) $\to$ AX(state=7) 	\\	
		\textbf{Description:} From state six, the next transition must be to state seven. \\
		\vspace{12pt}
		
		\textbf{Property 2:} AG(state=7) $\to$ AX(state=4)\\ 
		\textbf{Description:} From state three, the next transition must be to state four. \\
		\vspace{12pt}
		
		\textbf{Property 3:} 	AG(state=3 * door) $\to$ AX(state=1)\\ 
		\textbf{Description:} If in state three and the door is open then the next state, is state one. \\
		\vspace{12pt}
		
		\textbf{Property 4:} AG(state=5 * reset) $\to$ AX(state=3) \\
		\textbf{Description:} If in state five and a reset is signaled then the next state, is state three.\\
		\vspace{12pt}
		
		\textbf{Property 5:} AG(state=4 * cook) $\to$ AX(state=4)  \\
		\textbf{Description:} If in state four and it is cooking then it will remain in state four.\\
		
		
	
	\subsubsection{Results}
		\begin{lstlisting}[caption=NuSMV Kripke Diagram Output, label=p2_2_out]	
-- specification (AG state = 6 -> AX state = 7)  is true
-- specification (AG state = 7 -> AX state = 4)  is true
-- specification (AG (state = 3 & open_door) -> AX state = 1)  is true
-- specification (AG (state = 5 & reset) -> AX state = 3)  is true
-- specification (AG (state = 4 & cook) -> AX state = 4)  is true
		\end{lstlisting}
	
	
	\subsubsection{Proofs}
	The properties mentioned above are explained in figure \ref{p2_2_p}.
	\begin{figure}[h]\centering
	\includegraphics[height=0.45\textwidth]{images/p2_2_proof.png}
		\caption{Oven State Diagram}
			\label{p2_2_p}
		\end{figure}


\begin{enumerate}
\item If in state six there is only one possible transition. When in state six, the “warm up” signal is asserted, and the next state will be state seven.
\item If in state seven there is only one possible transition. When in state seven, the “start cooking” signal is asserted, and the next state will be state four.
\item If in state three and the “door open” signal is asserted, then the next state will be state one.
\item If in state five and the “reset” signal is asserted, then the next state will be state three.
\item If in state four and the “cooking” signal is asserted, then the next state will be state four.
\end{enumerate}	



	
	

%\vspace{12pt}
%
%\begin{figure}[h]\centering
%\includegraphics[height=0.5\textwidth]{images/c1c3.PNG}
%	\caption{Miter Circuit of C1 and C3}
%		\label{c3}
%	\end{figure}

\section{Equivalence Checking} 
We performed equivalence checking by model checking on the sequential design from homework 2.
\subsection{Results}

The results from the command line after running \texttt{nuSMV -int p3.smv} are:
 \begin{lstlisting}[caption=NuSMV Interactive Output, label=p3_out]	
NuSMV > go
NuSMV > pick_state
NuSMV > print_reachable_states
######################################################################
system diameter: 4
reachable states: 9 (2^3.16993) out of 64 (2^6)
######################################################################
\end{lstlisting}

When running the simulation normally, we see that \texttt{AG(out=0)} is false:
 \begin{lstlisting}[caption=NuSMV Interactive Output, label=p3_out]	
-- specification AG !out  is false
-- as demonstrated by the following execution sequence
Trace Description: CTL Counterexample
Trace Type: Counterexample
-> State: 1.1 <-
  state = S0
  x = FALSE
  out = FALSE
-> State: 1.2 <-
  state = S10
  out = TRUE
\end{lstlisting}



%  
%\section{Results}
%Do we need to show any output from boolsat.com?
\end{document}